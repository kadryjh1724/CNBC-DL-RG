\documentclass[a4paper, 10pt]{article}
\linespread{1.33}
% ┌─────────────────────┐
% │     preamble.tex    │
% └─────────────────────┘

% ══════════ [1] Basic document settings ══════════
\usepackage{fullpage}
\usepackage{geometry}
\geometry{
    top = 2cm,
    bottom = 4.5cm,
    left = 2.5cm,
    right = 2.5cm
}
\usepackage{lastpage}

\usepackage{xcolor}
\usepackage{graphicx}
\usepackage{tikz}
\usepackage{pgfplots}

\usepackage{enumerate}
\usepackage{sectsty}
\subsectionfont{\color{blue}}
\usepackage{enumitem}
\usepackage{array}
\newcolumntype{P}[1]{>{\centering\arraybackslash}p{#1}}

\usepackage{fourier-orns}
\usetikzlibrary{decorations.text}
\pgfplotsset{compat = newest}
\newcommand{\seprule}{
    \vspace*{1.5em}
    \vspace{-8pt}\hrulefill
    \raisebox{-2.1pt}{\quad\decofourleft\decotwo\decofourright\quad}\hrulefill
    \vspace*{1.5em}
}

\usepackage{hyperref}
\hypersetup{
  colorlinks=true,
  linkcolor=blue,
  linkbordercolor={0 0 1},
  urlcolor=blue
}

\usepackage{fancyhdr}
\usepackage[]{mdframed}

\renewcommand{\thesubsection}{\S{} \arabic{section}.\arabic{subsection}}

% ══════════ [2] Math packages ══════════
\usepackage{amsmath}
\usepackage{amsthm}
\usepackage{amsfonts}
\usepackage{amssymb}
\usepackage{amscd}
\usepackage{mathrsfs}
\usepackage{cancel}

% ══════════ [3] Miscellaneous & Fonts ══════════
\setlength{\parindent}{0.0in}
\setlength{\parskip}{0.05in}
\renewcommand{\footrulewidth}{0.4pt}

\usepackage{mathpazo}
\usepackage{domitian}
\usepackage[T1]{fontenc}
\let\oldstylenums\oldstyle
\setmonofont[Scale = 0.8]{DejaVu Sans Mono}

% ══════════ [4] amsthm setup ══════════

\newtheorem{theorem}{Theorem}[section]
\newtheorem{corollary}[theorem]{Corollary}
\newtheorem{lemma}[theorem]{Lemma}

\theoremstyle{definition}
\newtheorem{definition}[theorem]{Definition}

\theoremstyle{remark}
\newtheorem*{remark}{Remark}

\theoremstyle{definition}
\newtheorem{obs}[theorem]{Observation}

\theoremstyle{definition}
\newtheorem{exercise}[theorem]{Exercise}
\newmdtheoremenv[innertopmargin = 8pt,
                 innerbottommargin = 10pt]{exer}[theorem]{Exercise}

\theoremstyle{definition}
\newtheorem{example}[theorem]{Example}
% ┌─────────────────────┐
% │   usercommand.tex   │
% └─────────────────────┘

% ══════════ [1] short-hand notations ══════════
\newcommand{\mbf}{\mathbf}
\newcommand{\mrm}{\mathrm}
\newcommand{\mca}{\mathcal}
\newcommand{\msc}{\mathsc}
\newcommand{\mbb}{\mathbb}
\newcommand{\msf}{\mathsf}

\newcommand{\tbf}{\textbf}
\newcommand{\tit}{\textit}

\newcommand{\eps}{\epsilon}
\newcommand{\kB}{k_{\mathrm{B}}}
\def\dbar{{\mathchar'26\mkern-12mu d}}
\newcommand{\contradiction}{\ensuremath{{\Rightarrow\mspace{-2mu}\Leftarrow}}}
\newcommand{\im}{\mathrm{im}\,}
\renewcommand{\d}{\mathrm{d}}

\renewcommand\qedsymbol{$\blacksquare$}

\newcommand\lecturenumber{12}
\newcommand\lecturedate{Jan 3, 2025}

\pagestyle{fancyplain}
\headheight 40pt
\lhead{Lecture \lecturenumber\\CNBC Deep Learning Subgroup}
\rhead{Riemannian Geometry for Deep Learning \\\lecturedate}
\cfoot{Fall 2024, SNU}
\rfoot{\small\thepage}
\headsep 1.5em

\begin{document}
\setcounter{section}{7}
\section{Lie Groups and Lie Algebras}

\subsection{Lie Groups}

\begin{definition}[Lie group]
    A \tbf{Lie group} $G$ is a differentiable manifold which is endowed with group operations
    \begin{align*}
        \bullet &\,:\, G \times G \rightarrow G, \quad (g_{1},g_{2}) \mapsto g_{1} \cdot g_{2} \\
        {}^{-1} &\,:\, G \rightarrow G, \quad g \mapsto g^{-1}
    \end{align*}
    are smooth($\mca{C}^{\infty}$).
\end{definition}

\begin{remark}
    The identity element of a Lie group is written as $e$.
\end{remark}

\begin{exer}
    Show that the following groups are Lie groups.
    \begin{itemize}
        \item[(1)] $(\mbb{R}^{+}, \times)$
        \item[(2)] $(\mbb{R}, +)$
        \item[(3)] $(\mbb{R}^{2}, +)$ with $(x_{1},y_{1}) + (x_{2},y_{2}) = (x_{1}+x_{2},y_{1}+y_{2})$
        \item[(4)] $S^{1} = \{ e^{i\theta} \,|\, \theta \in \mbb{R} \, (\mrm{mod} \,2\pi)\}$ with\footnote{We call this group $\mrm{U}(1)$.}
        \[ e^{i\theta}e^{i\varphi} = e^{i(\theta+\varphi)}, \; (e^{i\theta})^{-1} = e^{-i\theta} \]
    \end{itemize}
\end{exer}

\begin{proof}
    The proof is simple. Hence, it is left for the readers.
\end{proof}

\newpage

% ===== ===== ===== ===== ===== ===== ===== ===== ===== ===== ===== ===== ===== ===== 

\begin{example}
    Let's take a look at \tbf{matrix groups}.
    \begin{itemize}
        \item[(1)] The group of $n \times n$ invertible matrices is called the \tbf{general linear group}, $\mrm{GL}(n,\mbb{R})$ and $\mrm{GL}(n,\mbb{C})$.
        \item[(2)] If we add additional $\det M = +1$ constraint, it is called the \tbf{special linear group}.
        \begin{align*}
            \mrm{SL}(n, \mbb{R}) &= \{ M \in \mrm{GL}(n, \mbb{R}) \,|\, \det M = +1 \} \\
            \mrm{SL}(n, \mbb{C}) &= \{ M \in \mrm{GL}(n, \mbb{C}) \,|\, \det M = +1 \}
        \end{align*}
        \item[(3)] The groups of isometries (or \tit{rigid motions}) of $\mbb{R}^{n}$ are called \tbf{orthogonal groups} and \tbf{special orthogonal groups}.
        \begin{align*}
            \mrm{O}(n) &= \{ M \in \mrm{GL}(n, \mbb{R}) \,|\, MM^{\msf{T}} = M^{\msf{T}}M = I_{n} \} \\
            \mrm{SO}(n) &= \mrm{O}(n) \cap \mrm{SL}(n, \mbb{R})
        \end{align*}
        \item[(4)] In $\mbb{C}^{n}$, groups with the same analogy are called \tbf{unitary groups} and \tbf{special unitary groups}.
        \begin{align*}
            \mrm{U}(n) &= \{ M \in \mrm{GL}(n, \mbb{C}) \,|\, MM^{\dagger} = M^{\dagger}M = I_{n} \} \\
            \mrm{SU}(n) &= \mrm{U}(n) \cap \mrm{SL}(n, \mbb{C})
        \end{align*}
        \item[(5)] The Lorentz group is also a Lie group,
        \[ \mrm{O}(1,3) = \{ M \in \mrm{GL}(4, \mbb{R}) \,|\, M\eta M^{\msf{T}} = \eta \} \]
        where $\eta = \mrm{diag}(-1, 1, 1, 1)$ is the Minkowski metric.
    \end{itemize}
\end{example}

Although we do not prove this, but I state an important theorem here.
\begin{theorem}
    Every closed subgroup $H$ of a Lie group $G$ is a Lie subgroup.
\end{theorem}

\seprule

Let $G$ be a group and $H \leq G$. For $g \in G$, a left(right) \tbf{coset} is defined by
\[ gH = \{ gh \,|\, h \in H \}, \; Hg = \{ hg \,|\, h \in H \} \]
We say that $H$ is a \tbf{normal subgroup} of $G$ (or $H \trianglelefteq G$) if
\[ g \in G,\, h \in H \;\Longrightarrow\; ghg^{-1} \in H \]
In terms of cosets, if $N \trianglelefteq G$, $gN = Ng$ for $g \in G$. If $N$ is a normal subgroup of $G$, we call $G/N$, the set of \tit{cosets} of $N$, the \tbf{quotient group}.

\newpage

% ===== ===== ===== ===== ===== ===== ===== ===== ===== ===== ===== ===== ===== ===== 

Let's consider $\mbb{Z}$ and $2\mbb{Z}$ as examples. Here, $2\mbb{Z}$ is the set of even numbers.
\begin{align*}
    \mbb{Z} &= \{ \cdots,\, -2,\, -1,\, 0,\, 1,\, 2,\, \cdots \} \\
    2\mbb{Z} &= \{ \cdots,\, -4,\, -2,\, 0,\, 2,\, 4,\, \cdots \}
\end{align*}
What are the cosets of $2\mbb{Z}$? Take several elements from $\mbb{Z}$ to compute cosets. Then,
\begin{align*}
    0 + 2\mbb{Z} &= \{ \cdots,\, -4,\, -2,\, 0,\, 2,\, 4,\, \cdots \} \\
    1 + 2\mbb{Z} &= \{ \cdots,\, -3,\, -1,\, 1,\, 3,\, 5,\, \cdots \} \\
    2 + 2\mbb{Z} &= \{ \cdots,\, -2,\, 0,\, 2,\, 4,\, 6,\, \cdots \} = 0 + 2\mbb{Z} \\
    3 + 2\mbb{Z} &= \{ \cdots,\, -1,\, 1,\, 3,\, 5,\, 7,\, \cdots \} = 1 + 2\mbb{Z} 
\end{align*}
We can see that $\mbb{Z}$ is \tit{partitioned} by $0 + 2\mbb{Z}$ and $1 + 2\mbb{Z}$! So we say
\[ \boxed{\mbb{Z} / 2\mbb{Z} \simeq \{0, 1\}} \]

\seprule

\begin{definition}[Coset space]
    Let $G$ be a Lie group and $H$ be a Lie subgroup of $G$. Define an
    \begin{itemize}
        \item[-] \tbf{equivalence relation} $\sim$ by $g \sim g'$ if $\exists\,h\in H \;\text{s.t.}\; g'=gh$
        \item[-] \tbf{equivalence class} $[g] = \{ gh \,|\, h \in H \} = gH$.
    \end{itemize}
    Then the \tbf{coset space} $G/H$ is a manifold (not necessarily a Lie group) with $\dim G/H = \dim G - \dim H$.
\end{definition}

\begin{obs}
    $G/H$ is a Lie group if $H \trianglelefteq G$\footnote{In other words, for arbitrary $g \in  G$ and $h \in H$, $ghg^{-1} \in H$.}.
    \begin{itemize}
        \item[-] Take $gH,\, g'H \in G/H$. If the group structure is well-defined in $G/H$, the product $(gH)(g'H)$ must be independent of the choice of representatives.
        \item[-] Let $gh$ and $g'h'$ be the representatives of $gH$ and $g'H$, respectively. Then
        \[ ghg'h' = gg'h''h' \in (gg')H \]
        Since there exists $h'' \in H$ such that $h'' = (g')^{-1}h((g')^{-1})^{-1}$. Therefore, group multiplication is well-defined.
        \item[-] Let $gh$ and $g'h'$ be the representation of $gH$ and $(gH)^{-1}$. Then $(gH)(gH)^{-1} = eH$ should hold.
        \[ ghg'h' = gg'h''h' = \underbrace{gg'}_{e}\underbrace{h''h'}_{\in H} \quad \Longrightarrow \quad (gH)^{-1} = g^{-1}H \]
        Hence, inverse element is well-defined.
    \end{itemize}
\end{obs}

\newpage

% ===== ===== ===== ===== ===== ===== ===== ===== ===== ===== ===== ===== ===== ===== 

\subsection{Lie Algebras}

\begin{definition}[Translations]
    Let $a$ and $g$ be elements of a Lie group $G$. The \tbf{right-(left-)translation} of $g$ by $a$
    \[ R_{a} \,:\, G \rightarrow G, \quad R_{a}g = ga \]
    and
    \[ L_{a} \,:\, G \rightarrow G, \quad L_{a}g = ag \]
    are diffeomorphisms\footnote{why?}. Hence, these maps induce
    \[ R_{a\ast} \,:\, T_{g}G \rightarrow T_{ga}G, \quad L_{a\ast} \,:\, T_{g}G \rightarrow T_{ag}G \]
\end{definition}

\begin{remark}
    Both right- and left-translation are suitable for the further discussion, but we will mainly consider the left-translations.
\end{remark}

\begin{obs}
    The diffeomorphism $L_{a}$ takes the identity $e$ to the element $a$, and induces $L_{a\ast,e} \,:\, T_{e}G \rightarrow T_{a}G$. Hence, if we can describe $T_{e}G$ at identity, then $L_{a\ast}T_{e}G$ will give a description of the tangent space $T_{g}G$ at any point $g \in G$.
\end{obs}

\seprule

\begin{definition}[Left-invariant vector fields]
    Let $X$ be a vector field on a Lie group $G$. For any $g \in G$, because left-translation $L_{a} \,:\, G \rightarrow G$ is a diffeomorphism, the \tit{pushforward} $L_{a\ast}X$ is a well-defined vector field on $G$. We say that the vector field $X$ is \tbf{left-invariant} if
    \[ \boxed{L_{a\ast}X = X} \; \text{or} \; \boxed{L_{a\ast}X|_{g} = X|_{ag}} \]
\end{definition}

\begin{obs}
    A left-invariant vector field $X$ is completely determined by its value $X|_{e}$ at the identity, since
    \[ \boxed{X|_{g} = L_{g\ast}X|_{e}} \; \cdots \; (\ast) \]
    \tit{Conversely}, given a tangent vector $X|_{e} \in T_{e}G$, we can define a vector field $X$ on $G$ by $(\ast)$. So defined, the vector field $X$ is left-invariant.
    \begin{align*}
        L_{a\ast}X|_{g} &= L_{a\ast}L_{g\ast}X|_{e} = (L_{a} \circ L_{g})_{\ast} X|_{e} \quad (\because \; \text{Exercise 5.9}) \\
        &= (L_{ag})_{\ast}X|_{e} = X|_{ag}
    \end{align*}
    Thus, there is a \tbf{one-to-one-correspondence} ($X|_{e} \;\leftrightarrow\; X$)
    \[ T_{e}G \; \leftrightarrow \; \mathfrak{g} := \{ \text{left-invariant vector fields on } G \} \]
\end{obs}

\begin{remark}
    If $X|_{g} = L_{g\ast}X|_{e}$ for all $g \in G$, we call $X$ the \tit{left-invariant vector field on $G$ generated by $X|_{e}$}.
\end{remark}

\begin{remark}
    The map $T_{e}G \,:\, \mathfrak{g}$ defined by $V \mapsto X_{V}$ is an \tit{isomorphism}: $\dim G = \dim \mathfrak{g}$. Hence, $\mathfrak{g}$ is a vector space isomorphic to $T_{e}G$.
\end{remark}

\newpage

% ===== ===== ===== ===== ===== ===== ===== ===== ===== ===== ===== ===== ===== ===== 

\begin{exer}
    Verify that a left-invariant vector field $X$ satisfies
    \begin{align*}
        L_{a\ast}X|_{g} &= X^{\mu}(g)\frac{\partial x^{\nu}(ag)}{\partial x^{\mu}(g)} \partial_{\nu}|_{ag} \quad (\because \; \text{definition of pushforward}) \\
        &= X^{\nu}(ag)\partial_{\nu}|_{ag}
    \end{align*}
\end{exer}

\begin{proof}
    The exercise itself is the proof.
\end{proof}

\begin{obs}
    Since $\mathfrak{g}$ is a set of vector fields, $\mathfrak{g} \subseteq \mathscr{X}(G)$. Let's show that $\mathfrak{g}$ is closed under the \tit{Lie bracket}. Take two points $g$ and $ag = L_{a}g \in G$. Then,
    \[ L_{a\ast}[X,Y]|_{g} = [L_{a\ast}X|_{g}, L_{a\ast}Y|_{g}] = [X,Y]|_{ag} \]
    by \tbf{Exercise 6.10(b)}. Hence, $[X,Y]$ is an another left-invariant vector field: $[X,Y] \in \mathfrak{g}$.
\end{obs}

\begin{definition}[Lie algebra]
    The set of left-invariant vector fields $\mathfrak{g}$ with the Lie bracket
    \[ [\;,\;] \,:\, \mathfrak{g} \times \mathfrak{g} \rightarrow \mathfrak{g} \]
    is called the \tbf{Lie algebra} of a Lie group $G$.
\end{definition}


\end{document}
