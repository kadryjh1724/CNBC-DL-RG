\documentclass[a4paper, 10pt]{article}
\linespread{1.33}
% ┌─────────────────────┐
% │     preamble.tex    │
% └─────────────────────┘

% ══════════ [1] Basic document settings ══════════
\usepackage{fullpage}
\usepackage{geometry}
\geometry{
    top = 2cm,
    bottom = 4.5cm,
    left = 2.5cm,
    right = 2.5cm
}
\usepackage{lastpage}

\usepackage{xcolor}
\usepackage{graphicx}
\usepackage{tikz}
\usepackage{pgfplots}

\usepackage{enumerate}
\usepackage{sectsty}
\subsectionfont{\color{blue}}
\usepackage{enumitem}
\usepackage{array}
\newcolumntype{P}[1]{>{\centering\arraybackslash}p{#1}}

\usepackage{fourier-orns}
\usetikzlibrary{decorations.text}
\pgfplotsset{compat = newest}
\newcommand{\seprule}{
    \vspace*{1.5em}
    \vspace{-8pt}\hrulefill
    \raisebox{-2.1pt}{\quad\decofourleft\decotwo\decofourright\quad}\hrulefill
    \vspace*{1.5em}
}

\usepackage{hyperref}
\hypersetup{
  colorlinks=true,
  linkcolor=blue,
  linkbordercolor={0 0 1},
  urlcolor=blue
}

\usepackage{fancyhdr}
\usepackage[]{mdframed}

\renewcommand{\thesubsection}{\S{} \arabic{section}.\arabic{subsection}}

% ══════════ [2] Math packages ══════════
\usepackage{amsmath}
\usepackage{amsthm}
\usepackage{amsfonts}
\usepackage{amssymb}
\usepackage{amscd}
\usepackage{mathrsfs}
\usepackage{cancel}

% ══════════ [3] Miscellaneous & Fonts ══════════
\setlength{\parindent}{0.0in}
\setlength{\parskip}{0.05in}
\renewcommand{\footrulewidth}{0.4pt}

\usepackage{mathpazo}
\usepackage{domitian}
\usepackage[T1]{fontenc}
\let\oldstylenums\oldstyle
\setmonofont[Scale = 0.8]{DejaVu Sans Mono}

% ══════════ [4] amsthm setup ══════════

\newtheorem{theorem}{Theorem}[section]
\newtheorem{corollary}[theorem]{Corollary}
\newtheorem{lemma}[theorem]{Lemma}

\theoremstyle{definition}
\newtheorem{definition}[theorem]{Definition}

\theoremstyle{remark}
\newtheorem*{remark}{Remark}

\theoremstyle{definition}
\newtheorem{obs}[theorem]{Observation}

\theoremstyle{definition}
\newtheorem{exercise}[theorem]{Exercise}
\newmdtheoremenv[innertopmargin = 8pt,
                 innerbottommargin = 10pt]{exer}[theorem]{Exercise}

\theoremstyle{definition}
\newtheorem{example}[theorem]{Example}
% ┌─────────────────────┐
% │   usercommand.tex   │
% └─────────────────────┘

% ══════════ [1] short-hand notations ══════════
\newcommand{\mbf}{\mathbf}
\newcommand{\mrm}{\mathrm}
\newcommand{\mca}{\mathcal}
\newcommand{\msc}{\mathsc}
\newcommand{\mbb}{\mathbb}
\newcommand{\msf}{\mathsf}

\newcommand{\tbf}{\textbf}
\newcommand{\tit}{\textit}

\newcommand{\eps}{\epsilon}
\newcommand{\kB}{k_{\mathrm{B}}}
\def\dbar{{\mathchar'26\mkern-12mu d}}
\newcommand{\contradiction}{\ensuremath{{\Rightarrow\mspace{-2mu}\Leftarrow}}}
\newcommand{\im}{\mathrm{im}\,}
\renewcommand{\d}{\mathrm{d}}

\renewcommand\qedsymbol{$\blacksquare$}

\newcommand\lecturenumber{09}
\newcommand\lecturedate{Dec 6, 2024}

\pagestyle{fancyplain}
\headheight 40pt
\lhead{Lecture \lecturenumber\\CNBC Deep Learning Subgroup}
\rhead{Riemannian Geometry for Deep Learning \\\lecturedate}
\cfoot{Fall 2024, SNU}
\rfoot{\small\thepage}
\headsep 1.5em

\begin{document}
\setcounter{section}{6}
\section{Differential Forms}

\subsection{Differential Forms: Basic Definitions}

\begin{definition}[Permutation]
    Let $\omega \in \mathfrak{T}^{(0,r)}_{p}(\mca{M})$. For a permutation $P \in S_{r}$\footnote{Here, $S_{r}$ denotes the \tit{symmetric group of order} $r$.} and $V_{1},\,V_{2},\,\cdots,\,V_{r} \in T_{p}\mca{M}$,
    \[ P\omega(V_{1},V_{2},\cdots,V_{r}) \equiv \omega(V_{P(1)},V_{P(2)},\cdots,V_{P(r)}) \]
    The component of $P\omega$ is given by
    \[ P\omega(e_{\mu_{1}},e_{\mu_{2}},\cdots,e_{\mu_{r}}) = \omega_{\mu_{P(1)}\mu_{P(2)}\cdots\mu_{P(r)}} \]
\end{definition}

\begin{definition}[Symmetrizer and antisymmetrizer]
    \hphantom{.}
    \begin{itemize}
        \item[(1)] The \tbf{symmetrizer} $\mca{S}$ is defined by 
        \[ \mca{S}\omega \equiv \frac{1}{r!}\sum_{P \in S_{r}} P\omega \]
        Note that $\mca{S}\omega$ is \tbf{totally symmetric}: $P\mca{S}\omega = \mca{S}\omega$ for arbitrary $P \in S_{r}$.
        \item[(2)] The \tbf{antisymmetrizer} $\mca{A}$ is defined by 
        \[ \mca{A}\omega \equiv \frac{1}{r!}\sum_{P \in S_{r}} \mrm{sgn}(P)P\omega \]
        where $\mrm{sgn}(P) = +1$ for even permutations and $\mrm{sgn}(P) = -1$ for odd permutations. Note that $\mca{A}\omega$ is \tbf{totally antisymmetric}: $P\mca{A}\omega = \mrm{sgn}(P)\mca{A}\omega$ for arbitrary $P \in S_{r}$.
    \end{itemize}
\end{definition}

\seprule

\begin{definition}[Differential forms]
    A \tbf{differential form of order} $r$ (or an $r$-\tbf{form}) is a totally antisymmetric tensor of type $(0,r)$. The vector space of $r$-forms in $p \in \mca{M}$ is denoted by $\Omega_{p}^{r}(\mca{M})$.
\end{definition}
\begin{remark}
    One-form like $\d{x}^{\mu}$ is a differential form of order 1.
\end{remark}

Then, how can we construct $r$-forms from $r$ one-forms?

\begin{definition}[Wedge product]
    The \tbf{wedge product} $\wedge$ of $r$ one-forms is defined by the totally antisymmetric tensor product.
    \[ \d{x}^{\mu_{1}} \wedge \d{x}^{\mu_{2}} \wedge \cdots \wedge \d{x}^{\mu_{r}} \equiv \sum_{P \in S_{r}} \mrm{sgn}(P) \,\d{x}^{\mu_{P(1)}} \otimes \cdots \otimes \d{x}^{\mu_{P(r)}} \]
\end{definition}
\newpage

% ===== ===== ===== ===== ===== ===== ===== ===== ===== ===== ===== ===== ===== ===== 

\begin{example}
    Let's construct 2-form and 3-form from the one-form $\d{x}^{\mu}$.
    \begin{itemize}
        \item[(1)] Since $|S_{2}| = 2$, two terms emerge.
        \[ \d{x}^{\mu} \wedge \d{x}^{\nu} = \d{x}^{\mu} \otimes \d{x}^{\nu} - \d{x}^{\nu} \otimes \d{x}^{\mu} \]
        \item[(2)] Since $|S_{3}| = 6$,
        \begin{align*}
            \d{x}^{\lambda} \wedge \d{x}^{\mu} \wedge \d{x}^{\nu} &= \d{x}^{\lambda} \otimes \d{x}^{\mu} \otimes \d{x}^{\nu} - \d{x}^{\lambda} \otimes \d{x}^{\nu} \otimes \d{x}^{\mu} \\
            & \quad+ \d{x}^{\mu} \otimes \d{x}^{\nu} \otimes \d{x}^{\lambda} - \d{x}^{\mu} \otimes \d{x}^{\lambda} \otimes \d{x}^{\nu} \\
            & \quad+ \d{x}^{\nu} \otimes \d{x}^{\lambda} \otimes \d{x}^{\mu} - \d{x}^{\nu} \otimes \d{x}^{\mu} \otimes \d{x}^{\lambda}
        \end{align*}
    \end{itemize}
\end{example}

\begin{obs}
    The wedge product satisfies:
    \begin{itemize}
        \item[(1)] $\d{x}^{\mu_{1}} \wedge \cdots \wedge \d{x}^{\mu_{r}} = 0$ if some index $\mu_{i}$ appears at least twice.
        \item[(2)] $\d{x}^{\mu_{1}} \wedge \cdots \wedge \d{x}^{\mu_{r}} = \mrm{sgn}(P)\d{x}^{\mu_{P(1)}} \wedge \cdots \wedge \d{x}^{\mu_{P(r)}}$ (the constructed $r$-form is totally antisymmetric).
        \item[(3)] $\d{x}^{\mu_{1}} \wedge \cdots \wedge \d{x}^{\mu_{r}}$ is linear in each $\d{x}^{\mu}$ (since $r$-forms are $(0,r)$-tensors).
    \end{itemize}
\end{obs}

\seprule

\begin{obs}
    The set of $r$-forms $\d{x}^{\mu_{1}} \wedge \cdots \wedge \d{x}^{\mu_{r}}$ forms a basis of $\Omega_{p}^{r}(\mca{M})$ and for $\omega \in \Omega_{p}^{r}(\mca{M})$,
    \begin{align*}
        \omega &= \omega_{\mu_{1}\cdots\mu_{r}}\,\d{x}^{\mu_{1}}\otimes\d{x}^{\mu_{2}}\otimes\cdots\otimes\d{x}^{\mu_{r}} \\
        &= \textcolor{red}{\boxed{\frac{1}{r!}\omega_{\mu_{1}\cdots\mu_{r}}\,\d{x}^{\mu_{1}}\wedge\d{x}^{\mu_{2}}\wedge\cdots\wedge\d{x}^{\mu_{r}}}}
    \end{align*}
\end{obs}

\begin{obs}
    Consider a (0,2)-tensor $\omega_{\mu\nu}$. This tensor can be decomposed into symmetric and antisymmetric parts: the symmetric part
    \[ \sigma_{\mu\nu} = \frac{\omega_{\mu\nu} + \omega_{\nu\mu}}{2} \quad\rightarrow\quad \sigma_{\mu\nu}\,\d{x}^{\mu}\wedge\d{x}^{\nu} = 0 \]
    and antisymmetric part
    \[ \alpha_{\mu\nu} = \frac{\omega_{\mu\nu} - \omega_{\nu\mu}}{2} \quad\rightarrow\quad \alpha_{\mu\nu}\,\d{x}^{\mu}\wedge\d{x}^{\nu} = \frac{\omega_{\mu\nu} - \omega_{\nu\mu}}{2}\,\d{x}^{\mu}\wedge\d{x}^{\nu} = \omega_{\mu\nu}\,\d{x}^{\mu}\wedge\d{x}^{\nu} \]
    Observe that only antisymmetric part of $\omega_{\mu\nu}$ can contribute to two-form.
\end{obs}
\newpage

% ===== ===== ===== ===== ===== ===== ===== ===== ===== ===== ===== ===== ===== ===== 

\begin{obs}
    Since there are $\binom{m}{r}$ choices in the set $(\mu_{1},\cdots,\mu_{r})$ from $(1,\cdots,m)$,
    \[ \dim\Omega_{p}^{r}(\mca{M}) = \binom{m}{r} \]
    \begin{itemize}
        \item[-] Define $\Omega_{p}^{0}(\mca{M}) \equiv \mbb{R}$.
        \item[-] $\Omega_{p}^{1}(\mca{M}) = T_{p}^{\ast}\mca{M}$ ($\dim T_{p}^{\ast}\mca{M} = m)$.
        \item[-] If $r \ngeq m$, some index appears at least twice in the antisymmetrical sum, so such differential form vanishes.
        \item[-] Since $\binom{m}{r} = \binom{m}{m-r}$, $\dim\Omega_{p}^{r}(\mca{M}) = \dim\Omega_{p}^{m-r}(\mca{M})$.
        \item[-] Moreover, since $\Omega_{p}^{r}(\mca{M})$ is a vector space, $\Omega_{p}^{r}(\mca{M}) \simeq \Omega_{p}^{m-r}(\mca{M})$.
    \end{itemize}
\end{obs}

Until now, we used \tit{wedge product} to construct $r$-forms. Can we make higher-order forms by combining $q$-forms and $r$-forms? Yes.
\begin{definition}
    The \tbf{exterior product} of a $q$-form and a $r$-form
    \[ \wedge \,:\, \Omega_{p}^{q}(\mca{M}) \times \Omega_{p}^{r}(\mca{M}) \rightarrow \Omega_{p}^{q+r}(\mca{M}) \]
    is defined by a trivial extension. For $\omega \in \Omega_{p}^{q}(\mca{M})$ and $\xi \in \Omega_{p}^{r}(\mca{M})$, the action of the $(q+r)$-form $\omega\wedge\xi$ on $(q+r)$ vectors is defined by
    \[ (\omega\wedge\xi)(V_{1},\cdots,V_{q+r}) = \frac{1}{q!r!}\sum_{P \in S_{q+r}} \mrm{sgn}(P)\omega(V_{P(1)},\cdots,V_{P(q)})\xi(V_{P(q+1)},\cdots,V_{P(q+r)}) \]
\end{definition}

\begin{remark}
    If $q+r > m$, $\omega\wedge\xi$ vanishes naturally.
\end{remark}
\begin{remark}
    With this exterior product, we define an algebra
    \[ \Omega_{p}^{\ast}(\mca{M}) \equiv \Omega_{p}^{0}(\mca{M}) \oplus \Omega_{p}^{1}(\mca{M}) \oplus \cdots \oplus \Omega_{p}^{m}(\mca{M}) \]
    $\Omega_{p}^{\ast}(\mca{M})$ denotes the space of all differential forms at $p$. This space is closed under the exterior product. Moreover, we may assign an $r$-form \tit{smoothly} at each point on a manifold $\mca{M}$. We denote the space of smooth $r$-forms on $\mca{M}$ by $\Omega^{r}(\mca{M})$.
\end{remark}

\begin{table}[htbp]
    \begin{center}
        \begin{tabular}{ccc}
            \hline
            $r$-form & Basis & Dimension \\
            \hline
            $\Omega^{0}(\mca{M}) = \mca{F}(\mca{M})$ & $\{1\}$ & 1 \\
            $\Omega^{1}(\mca{M}) = T^{\ast}\mca{M}$ & $\{\d{x}^{\mu}\}$ & $m$ \\
            $\Omega^{2}(\mca{M})$ & $\{\d{x}^{\mu_{1}}\wedge\d{x}^{\mu_{2}}\}$ & $m(m-1)/2!$ \\
            $\Omega^{3}(\mca{M})$ & $\{\d{x}^{\mu_{1}}\wedge\d{x}^{\mu_{2}}\wedge\d{x}^{\mu_{3}}\}$ & $m(m-1)(m-2)/3!$ \\
            \vdots & \vdots & \vdots \\
            $\Omega^{m}(\mca{M})$ & $\{\d{x}^{1}\wedge\cdots\wedge\d{x}^{m}\}$ & 1 \\
            \hline
        \end{tabular}
        \caption{The space of smooth $r$-forms.}
    \end{center}
\end{table}
\newpage

% ===== ===== ===== ===== ===== ===== ===== ===== ===== ===== ===== ===== ===== ===== 

\begin{exer}
    Consider a Cartesian coordinates $(x,y)$ in $\mbb{R}^{2}$. Show that
    \[ \d{x} \wedge \d{y} = r\,\d{r} \wedge \d{\theta} \]
\end{exer}

\begin{proof}
    \begin{align*}
        \d{x} \wedge \d{y} &= \d{x} \otimes \d{y} - \d{y} \otimes \d{x} \\
        &= (\cos\theta\,\d{r} - r\sin\theta\,\d{\theta}) \otimes (\sin\theta\,\d{r} + r\cos\theta\,\d{\theta}) \\
        & \qquad\qquad - (\sin\theta\,\d{r} + r\cos\theta\,\d{\theta}) \otimes (\cos\theta\,\d{r} - r\sin\theta\,\d{\theta}) \\
        &= r\,d{r} \otimes \d\theta - r\,\d\theta \otimes \d{r} = r\,\d{r}\wedge\d\theta
    \end{align*}
\end{proof}

We conclude this lecture by proving several important properties of exterior products. Before we do that, we prove this useful lemma first.

\begin{lemma}
    Let $\xi \in \Omega_{p}^{q}(\mca{M})$ and $\eta \in \Omega_{p}^{r}(\mca{M})$. Then
    \[ \mca{A}(\mca{A}(\xi) \otimes \eta) = \mca{A}(\xi \otimes \eta) \]
\end{lemma}

\begin{proof}
    By definition,
    \[ \mca{A}(\mca{A}(\xi) \otimes \eta) = \frac{1}{(q+r)!}\sum_{\sigma\in S_{q+r}}\mrm{sgn}(\sigma)\sigma\left(\frac{1}{q!}\sum_{\tau \in S_{q}} \mrm{sgn}( \tau)\tau\xi \otimes \eta\right) \]
    If we interpret $\tau \in S_{q}$ as a permutation in $S_{q+r}$ such that $\tau(i) = i \;(i = q+1, \cdots, q+r)$, $\tau\xi \otimes \eta = \tau(\xi\otimes\eta)$.
    \[ \phantom{\mca{A}(\mca{A}(\xi) \otimes \eta)} = \frac{1}{q!(q+r)!}\sum_{\sigma \in S_{q+r}}\sum_{\tau \in S_{q}}\mrm{sgn}(\sigma)\mrm{sgn}(\tau)(\sigma\tau)(\xi\otimes\eta) \]
    Let $\mu = \sigma\tau \in S_{q+r}$. For each $\mu$, there are $q!$ ways to write $\mu = \sigma\tau$ with $\sigma \in S_{q+r}$ and $\tau \in S_{q}$, because each $\tau \in S_{q}$ determines a unique $\sigma$ by $\sigma = \mu\tau^{-1}$. Hence,
    \[ \phantom{\mca{A}(\mca{A}(\xi) \otimes \eta)} = q! \cdot \frac{1}{q!(q+r)!}\sum_{\mu \in S_{q+r}}\mrm{sgn}(\mu)\mu(\xi\otimes\eta) = \mca{A}(\xi\otimes\eta) \]
\end{proof}
\newpage

% ===== ===== ===== ===== ===== ===== ===== ===== ===== ===== ===== ===== ===== ===== 

\begin{exer}
    Let $\xi \in \Omega_{p}^{q}(\mca{M}),\,\eta \in \Omega_{p}^{r}(\mca{M})$ and $\omega \in \Omega_{p}^{s}(\mca{M})$. Show that
    \begin{itemize}
        \item[(1)] $\xi \wedge \xi = 0$ if $q$ is odd.
        \item[(2)] $\xi \wedge \eta = (-1)^{qr}\eta \wedge \xi$.
        \item[(3)] $(\xi\wedge\eta)\wedge\omega = \xi\wedge(\eta\wedge\omega)$.
    \end{itemize}
\end{exer}

\begin{proof}
    \hphantom{.}
    \begin{itemize}
        \item[(1)] Let $V_{1},\cdots,V_{2q} \in T_{p}\mca{M}$.
        \begin{align*}
            (\xi\wedge\xi)(V_{1},\cdots,V_{2q}) &= \frac{1}{(q!)^{2}}\sum_{P \in S_{2q}}\mrm{sgn}(P)\xi(V_{P(1)},\cdots,V_{P(q)})\xi(V_{P(q+1)},\cdots,V_{P(2q)}) \\
            &= -\frac{1}{(q!)^{2}}\sum_{P \in S_{2q}}\mrm{sgn}(P)\xi(V_{P(q+1)},\cdots,V_{P(2q)})\xi(V_{P(1)},\cdots,V_{P(q)}) = 0
        \end{align*}
        Here, changing permutation 
        \[ (P(1),\cdots,P(q),P(q+1),\cdots,P(2q)) \]
        to
        \[ (P(q+1),\cdots,P(2q),P(1),\cdots,P(q)) \]
        requires $q^{2}$ swaps. Hence, if $q$ is odd, additional (-1) factor arises.
        \item[(2)] In the exactly same manner, you can show that $(-1)^{qr}$ factor arises from changing the order of two forms.
        \item[(3)] By definition,
        \begin{align*}
            (\xi \wedge \eta) \wedge \omega) &= \frac{(q+r+s)!}{(q+r)!s!} \mca{A}((\xi\wedge\eta) \otimes \omega) \\
            &= \frac{(q+r+s)!}{(q+r)!s!} \frac{(q+r)!}{q!r!} \mca{A}(\mca{A}(\xi \otimes \eta) \otimes \omega) \\
            &= \frac{(q+r+s)!}{q!r!s!}\mca{A}(\xi \otimes \eta \otimes \omega) \quad (\because \text{\textbf{Lemma 7.12}})
        \end{align*}
        You can yield same expression for $\xi \wedge (\eta \wedge \omega)$.
    \end{itemize}
\end{proof}

\end{document}
