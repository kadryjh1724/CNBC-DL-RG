\documentclass[a4paper, 10pt]{article}
\linespread{1.33}
% ┌─────────────────────┐
% │     preamble.tex    │
% └─────────────────────┘

% ══════════ [1] Basic document settings ══════════
\usepackage{fullpage}
\usepackage{geometry}
\geometry{
    top = 2cm,
    bottom = 4.5cm,
    left = 2.5cm,
    right = 2.5cm
}
\usepackage{lastpage}

\usepackage{xcolor}
\usepackage{graphicx}
\usepackage{tikz}
\usepackage{pgfplots}

\usepackage{enumerate}
\usepackage{sectsty}
\subsectionfont{\color{blue}}
\usepackage{enumitem}
\usepackage{array}
\newcolumntype{P}[1]{>{\centering\arraybackslash}p{#1}}

\usepackage{fourier-orns}
\usetikzlibrary{decorations.text}
\pgfplotsset{compat = newest}
\newcommand{\seprule}{
    \vspace*{1.5em}
    \vspace{-8pt}\hrulefill
    \raisebox{-2.1pt}{\quad\decofourleft\decotwo\decofourright\quad}\hrulefill
    \vspace*{1.5em}
}

\usepackage{hyperref}
\hypersetup{
  colorlinks=true,
  linkcolor=blue,
  linkbordercolor={0 0 1},
  urlcolor=blue
}

\usepackage{fancyhdr}
\usepackage[]{mdframed}

\renewcommand{\thesubsection}{\S{} \arabic{section}.\arabic{subsection}}

% ══════════ [2] Math packages ══════════
\usepackage{amsmath}
\usepackage{amsthm}
\usepackage{amsfonts}
\usepackage{amssymb}
\usepackage{amscd}
\usepackage{mathrsfs}
\usepackage{cancel}

% ══════════ [3] Miscellaneous & Fonts ══════════
\setlength{\parindent}{0.0in}
\setlength{\parskip}{0.05in}
\renewcommand{\footrulewidth}{0.4pt}

\usepackage{mathpazo}
\usepackage{domitian}
\usepackage[T1]{fontenc}
\let\oldstylenums\oldstyle
\setmonofont[Scale = 0.8]{DejaVu Sans Mono}

% ══════════ [4] amsthm setup ══════════

\newtheorem{theorem}{Theorem}[section]
\newtheorem{corollary}[theorem]{Corollary}
\newtheorem{lemma}[theorem]{Lemma}

\theoremstyle{definition}
\newtheorem{definition}[theorem]{Definition}

\theoremstyle{remark}
\newtheorem*{remark}{Remark}

\theoremstyle{definition}
\newtheorem{obs}[theorem]{Observation}

\theoremstyle{definition}
\newtheorem{exercise}[theorem]{Exercise}
\newmdtheoremenv[innertopmargin = 8pt,
                 innerbottommargin = 10pt]{exer}[theorem]{Exercise}

\theoremstyle{definition}
\newtheorem{example}[theorem]{Example}
% ┌─────────────────────┐
% │   usercommand.tex   │
% └─────────────────────┘

% ══════════ [1] short-hand notations ══════════
\newcommand{\mbf}{\mathbf}
\newcommand{\mrm}{\mathrm}
\newcommand{\mca}{\mathcal}
\newcommand{\msc}{\mathsc}
\newcommand{\mbb}{\mathbb}
\newcommand{\msf}{\mathsf}

\newcommand{\tbf}{\textbf}
\newcommand{\tit}{\textit}

\newcommand{\eps}{\epsilon}
\newcommand{\kB}{k_{\mathrm{B}}}
\def\dbar{{\mathchar'26\mkern-12mu d}}
\newcommand{\contradiction}{\ensuremath{{\Rightarrow\mspace{-2mu}\Leftarrow}}}
\newcommand{\im}{\mathrm{im}\,}
\renewcommand{\d}{\mathrm{d}}

\renewcommand\qedsymbol{$\blacksquare$}

\newcommand\lecturenumber{01}
\newcommand\lecturedate{Oct 11, 2024}

\pagestyle{fancyplain}
\headheight 40pt
\lhead{Lecture \lecturenumber\\CNBC Deep Learning Subgroup}
\rhead{Riemannian Geometry for Deep Learning \\ Oct 11, 2024}
\cfoot{Fall 2024, SNU}
\rfoot{\small\thepage}
\headsep 1.5em

\begin{document}

\section{Vectors and Vector Spaces}

\subsection{Vector Space: Definitions}

\begin{definition}[Vector spaces]
    Let $F$ be a field (such as $\mbb{R}$, $\mbb{C}$, etc.). Suppose that there is a set $V$ equipped with addition $+ \,:\, V \times V \rightarrow V$ and scalar multiplication $\cdot \,:\, F \times V \rightarrow V$. If $V$ satisfies the following 8 properties (for $a, b \in F$ and $u, v, w \in V$),
    \begin{itemize}
        \item[(1)] $(u + v) + w = u + (v + w)$
        \item[(2)] $v + w = w + v$
        \item[(3)] $\exists \, 0 \in V$ such that $v + 0 = 0 + v = v$
        \item[(4)] For given $v \in V$, $\exists \, -v \in V$ such that $v + (-v) = 0$
        \item[(5)] $(a + b)v = av + bv$
        \item[(6)] $a(v + w) = av + aw$
        \item[(7)] $a(bv) = (ab)v$
        \item[(8)] $1 \cdot v = v$
    \end{itemize}
    then $V$ is a \tbf{vector space} over $F$. Elements of $F$ and $V$ are called \tbf{scalars} and \tbf{vectors}, respectively.
\end{definition}

Sets such as $F^{n}$ (or something like $\mbb{R}^{n}$) are familiar example of vector spaces.

\seprule

\begin{exer}
    Prove the following.
    \begin{itemize}
        \item[(1)] The \tbf{identity element} $0 \in V$ exists uniquely.
        \item[(2)] For $v \in V$, \tbf{inverse element} $-v \in V$ exists uniquely.
    \end{itemize}
\end{exer}
\begin{proof}
    \begin{itemize}
        \item[(1)] Let $0'$ be another identity element. Then
        \begin{align*}
            0 &= 0 + 0' \quad (0' \, \text{is an identity}) \\
            &= 0' \quad (0 \, \text{is an identity})
        \end{align*}
        \item[(2)] Let $w$ be another inverse element ($v + w = 0$). Then
        \[ -v = -v + 0 = -v + (v + w) = (-v + v) + w = 0 + w = w \]
    \end{itemize}
\end{proof}
\newpage

% ===== ===== ===== ===== ===== ===== ===== ===== ===== ===== ===== =====

\begin{exer}
    Prove the following. $(v \in V, a \in F)$
    \begin{itemize}
        \item[(1)] $0v = 0$
        \item[(2)] $a0 = 0$
        \item[(3)] $-v = (-1)v$
        \item[(4)] If $v \neq 0$ and $a \neq 0$, then $av \neq 0$.
    \end{itemize}
\end{exer}
\begin{proof}
    \begin{itemize}
        \item[(1)] $0v = (0 + 0)v = 0v + 0v \;\Longrightarrow\; 0v = 0$
        \item[(2)] $a0 = a(0 + 0) = a0 + a0 \;\Longrightarrow\; a0 = 0$
        \item[(3)] $v + (-1)v = 1v + (-1)v = (1 - 1)v = 0v = 0 \;\Longrightarrow\; -v = (-1)v$
        \item[(4)] Suppose that $av = 0$. Then
        \[ v = 1v = \left(\frac{1}{a} \cdot a\right)v = \frac{1}{a}(av) = \frac{1}{a}0 = 0 \quad \contradiction \]
    \end{itemize}
\end{proof}
\begin{definition}[Subspaces]
    Let $W$ be a subset of $F$-vector space $V$. If $W$ becomes a $F$-vector space for the addition and scalar multiplication operations inherited from $V$, $W$ is a \tbf{subspace} of $V$. We denote this by $W \leq V$.
\end{definition}
\begin{obs}
    For $\varnothing \neq W \subseteq V$, $W \leq V$ if and only if
    \begin{itemize}
        \item[(1)] $w_{1}, w_{2} \in W \;\Longrightarrow\; w_{1} + w_{2} \in W$
        \item[(2)] $a \in F, \, w \in W \;\Longrightarrow\; aw \in W$
    \end{itemize}
\end{obs}
The proof is almost trivial (you may check it by yourself). In short, $W$ becomes a subspace of $V$ if it is closed for the addition and scalar multiplication.

\seprule

Lastly, we introduce an extremely important concept.
\begin{definition}[Isomorphisms]
    Let $V$ and $V'$ be $F$-vector spaces. If there exists a bijection $\varphi \,:\, V \rightarrow V'$ such that
    \[ \varphi(v + w) = \varphi(v) + \varphi(w), \quad \varphi(av) = a\varphi(v) \quad (a \in F, \; v, w \in V) \]
    then $V$ and $V'$ are \tbf{isomorphic} ($V \simeq V')$. The bijection $\varphi$ is called an \tbf{isomorphism}.
\end{definition}
\begin{remark}
    Isomorphism $\varphi$ preserves the addition and scalar multiplication, as you can see in its definition. Moreover, if there is an isomorphism between two vector spaces, those two spaces are \tbf{identical} \tit{de facto}.
\end{remark}
\newpage

% ===== ===== ===== ===== ===== ===== ===== ===== ===== ===== ===== =====

\begin{definition}{Basis of a vector space}
    A non-empty subset of $V$, $\mathfrak{B} \subseteq V$ is called a \tbf{basis} of $V$ if it satisfies the following properties.
    \begin{itemize}
        \item[(1)] $\left< \mathfrak{B} \right> = V$, where $\left< \mathfrak{B} \right>$ denotes the set of all linear combinations of vectors in $\mathfrak{B}$.
        In other words, $\mathfrak{B}$ generates (or \tbf{spans}) $V$.
        \item[(2)] $\mathfrak{B}$ is linearly independent.
    \end{itemize}
\end{definition}
\begin{remark}
    When we say a set of vectors $v_{1}, \, \cdots, \, v_{n}$ are \tit{linearly independent}, this implies
    \[ \text{If} \, a_{1}v_{1} + a_{2}v_{2} + \cdots + a_{n}v_{n} = 0, \, \text{then all} \, a_{i} = 0 \; (i = 1, \, 2, \, \cdots, \, n) \]
\end{remark}
\begin{example}
    Consider $F = \mbb{R}$ and $V = \mbb{R}^{3}$. Then we have \tbf{standard (Euclidean) basis}
    \[ e_{1} = \begin{bmatrix}1\\ 0\\ 0\end{bmatrix}, \; e_{2} = \begin{bmatrix}0\\ 1\\ 0\end{bmatrix}, \; e_{3} = \begin{bmatrix}0\\ 0\\ 1\end{bmatrix} \]
    The basis is not unique; actually, there exists multiple different basis that spans $\mbb{R}^{3}$.
\end{example}

We often denote a vector in $\mbb{R}^{3}$ as $v = [5, 2, 3]^{\msf{T}}$. In fact, this means
\[ v = 5\begin{bmatrix}1\\ 0\\ 0\end{bmatrix} + 2\begin{bmatrix}0\\ 1\\ 0\end{bmatrix} + 3\begin{bmatrix}0\\ 0\\ 1\end{bmatrix} = 5e_{1} + 2e_{2} + 3e_{3} \]
where 5, 2 and 3 indicates the \tbf{components} of $v$.
\begin{definition}[Components of vectors]
    Let $\mathfrak{B} = \{ e_{1}, \,\cdots\, e_{n}\}$ be a $F$-basis of $V$. Then any vector $v \in V$ can be represented by the linear combination of $\{ e_{\mu} \}_{\mu = 1}^{n}$ (by definition).
    \[ v = v^{1}e_{1} + v^{2}e_{2} + \cdots + v^{n}e_{n} = \sum_{\mu = 1}^{n} v^{\mu} e_{\mu} \]
    We call these $v^{\mu}$'s as \tbf{components}.
\end{definition}
\begin{remark}
    Note that basis of a vector space, $e_{\mu}$, has lower index, while component of a vector $v^{\mu}$ has upper index.
\end{remark}
\begin{remark}
    In differential geometry, we often encounter summations on overlapping indices. 
    
    \tbf{Einstein notation} simplies this by (simply) omitting summation signs.
    \[ \sum_{\mu = 1}^{n} v^{\mu} e_{\mu} := v^{\mu} e_{\mu} \]
    If we encounter overlapping indices - one in superscript and one in subscript, we assume there is an omitted summation.
\end{remark}
\begin{definition}[Dimension]
    $\dim V = |\mathfrak{B}|$.
\end{definition}
\newpage

% ===== ===== ===== ===== ===== ===== ===== ===== ===== ===== ===== =====

\subsection{Linear Maps and Dual Vector Spaces}
\begin{definition}[Linear map]
    Let $V$ and $W$ be $F$-vector spaces. If a map $L \,:\, V \rightarrow W$ satisfies
    \[ L(v+w) = L(v) + L(w), \quad L(av) = a\cdot L(v) \quad (a \in F, \, v, w \in V) \]
    we call $L$ a \tbf{linear map}.
\end{definition}
\begin{exer}
    Let $L$ be a linear map and $u, v \in V$. Prove the followings.
    \begin{itemize}
        \item[(1)] $L(0) = 0$
        \item[(2)] $L(-v) = -L(v)$
        \item[(3)] $L(u-v) = L(u) - L(v)$
    \end{itemize}
\end{exer}
\begin{proof}
    \begin{itemize}
        \item[(1)] $L(0) = L(0+0) = L(0) + L(0) \;\Longrightarrow\; L(0) = 0$
        \item[(2)] $L(0) = L(v-v) = L(v) + L(-v) \;\Longrightarrow\; -L(v) = L(-v)$
        \item[(3)] $L(u-v) = L(u) + L(-v) = L(u) - L(v)$
    \end{itemize}
\end{proof}
While talking about maps, we should mention these two.
\begin{definition}[Kernel and image]
    Let $L \,:\, V \rightarrow W$ be a linear map.
    \begin{itemize}
        \item[(1)] The \tbf{kernel} of $L$ is defined by $\ker L = L^{-1}(0) = \{ v \in V \,|\, L(v) = 0 \}$.
        \item[(2)] The \tbf{image} of $L$ is defined by $\im L = L(V) = \{ L(V) \in W \,|\, v \in V \}$.
    \end{itemize}
\end{definition}
\begin{exer}
    Show that (1) $\ker L \leq V$ and (2) $\im L \leq W$.
\end{exer}
\begin{proof}
    By \tbf{Observation 1.3}, it suffices to show that $\ker L$ and $\im L$ are closed.
    \begin{itemize}
        \item[(1)] Let $v, w \in \ker L$. Then
        \[ L(v + w) = L(v) + L(w) = 0 + 0 = 0, \quad L(av) = aL(v) = 0 \; (a \in F) \]
        Hence, $v + w, av \in \ker L \;\Longrightarrow\; \ker L \leq V$.
        \item[(2)] Let $L(v), L(w) \in \im L$. Then
        \[ L(v) + L(w) = L(\underbrace{v+w}_{\in V}) \in \im L, \quad aL(v) = L(\underbrace{av}_{\in V}) \in \im L \; (a \in F) \]
        gives $\im L \leq W$.
    \end{itemize}
\end{proof}
\newpage

% ===== ===== ===== ===== ===== ===== ===== ===== ===== ===== ===== =====

\begin{theorem}[Dimension theorem]
    Let $V$ be a finite dimensional vector space and $L$ be a linear map $L \,:\, V \rightarrow W$. Then
    \[ \dim V = \dim \ker L + \dim \im L \]
\end{theorem}
\begin{proof}
    Let basis of $\ker L$ be $\{ e_{1} ,\, \cdots ,\, e_{r} \}$ (in other words, $\dim \ker L := r$).

    Since $\ker L \leq V$, we can \tit{extend} the basis of $\ker L$ to the basis of $V$\footnote{$\because$ Basis extension theorem, which we neither mentioned nor proved.} as following.
    \[ \mathfrak{B} = \{ e_{1} ,\, \cdots ,\, e_{r}, \textcolor{blue}{e_{r+1} ,\, \cdots ,\, e_{n}} \} \]
    Here, note that $L(e_{1}) = \cdots = L(e_{r})$ (they are elements of $\ker L$).

    \tbf{WTS}: $\{ L(e_{r+1}) ,\, \cdots ,\, L(e_{n}) \}$ is a basis of $\im L$.
    \begin{itemize}
        \item[(1)] $\{ L(e_{r+1}) ,\, \cdots ,\, L(e_{n}) \}$ are linearly independent. In the following equation, we expect all $a^{k} \; (k = r+1, \cdots, n)$ should be zero.
        \[ \sum_{k = r+1}^{n} a^{k} L(e_{k}) = 0 = L \left(\sum_{k = r+1}^{n} a^{k}e_{k}\right) \]
        The last equal sign holds since $L$ is a linear map. Since substituting $a^{k}e_{k}$\footnote{This is written in Einstein notation.} into $L$ yields zero, $a^{k}e_{k} \in \ker L$. Hence, $a^{k}e_{k}$ can be represented by the linear combination of $\{ e_{1} ,\, \cdots ,\, e_{r} \}$.
        \[ \sum_{k=r+1}^{n}a^{k}e_{k} = \sum_{k=1}^{r}b^{k}e_{k} \]
        This equation can be arranged into a single summation.
        \[ \sum_{k=1}^{n}c^{k}e_{k} = 0 \]
        where $c^{k} = b^{k}$ for $k = 1, \cdots, r$ and $c^{k} = -a^{k}$ for $k = r+1, \cdots, n$. Since $\{e_{k}\}_{k=1}^{n}$ is a basis of $V$, these vectors are linearly independent - hence, all $c^{k}$ are zero. Therefore, all $a^{k} \; (k = r+1, \cdots, n)$ becomes zero automatically.
        \item[(2)] $\left< L(e_{r+1}) ,\, \cdots ,\, L(e_{n}) \right> = \im L$. 
        
        This is straightforward; for arbitrary $L(v) \in \im L$,
        \begin{align*}
            L(v) &= L \left(\sum_{k=1}^{n} v^{k}e_{k}\right) \\
            &= L \left(\sum_{k=1}^{r} v^{k}e_{k} + \sum_{k=r+1}^{n} v^{k}e_{k}\right) \\
            &= \cancel{L \left(\sum_{k=1}^{r} v^{k}e_{k}\right)} + L \left(\sum_{k=r+1}^{n} v^{k}e_{k}\right) \quad (\because \sum_{k=1}^{r} v^{k}e_{k} \in \ker L) \\
            &= \sum_{k=r+1}^{n} v^{k} L(e_{k})
        \end{align*}
    \end{itemize}
\end{proof}
\newpage

% ===== ===== ===== ===== ===== ===== ===== ===== ===== ===== ===== =====

\begin{definition}[Vector space of linear maps]
    Let $\mathfrak{L}(V,W)$ be the set of all linear maps from $V$ to $W$.
    \[ \mathfrak{L}(V,W) = \{ L \,:\, V \rightarrow W \,|\, L \, \text{is linear} \} \]
    By defining addition and scalar multiplication of linear maps as following,
    \begin{align*}
        \text{addition} \,&: \quad (L+M)(v) := L(v) + M(v) \\
        \text{scalar multiplication} \,&: \quad (aL)(v) := aL(v)
    \end{align*}
    $(L, M \in \mathfrak{L}(V, W),\; v \in V,\; a \in F)$ $\mathfrak{L}(V,W)$ becomes a \tbf{vector space} of linear maps.
\end{definition}

\seprule

\begin{definition}[Dual vector space]
    Especially,
    \[ \mathfrak{L}(V, F) := V^{\ast} \]
    is called a \tbf{dual vector space}. The element of $V^{\ast}$, $w \in V^{\ast}$ is called a \tbf{dual vector} (or a \tbf{covector}).
\end{definition}
\begin{remark}
    An element of a dual vector space is actually a \tit{linear map}. In other words, for $w \in V^{\ast}$ and $v \in V$,
    \[ w \,:\, V \rightarrow F \quad \text{and} \quad w(v) \in F \]
\end{remark}
Now we prove an important theorem that helps us grasp concepts of dual basis and components of dual vectors.
\begin{theorem}
    $\dim V^{\ast} = \dim V$.
\end{theorem}
\begin{proof}
    Let $\mathfrak{B} = \{e_{1} ,\, \cdots ,\, e_{n} \} = \{ e_{\mu} \}$ be a basis of $V$ ($\dim V := n$).

    Define some linear maps (or dual vectors) $e^{\mu} \,:\, V \rightarrow F$ as
    \[ e^{\mu}(e_{\nu}) = \delta^{\mu}{}_{\nu} \]
    where $\delta^{\mu}{}_{\nu}$ is the usual Kroenecker delta\footnote{You may wonder why one index is superscripted while the other is subscripted. This will be clarified in the next week's lecture.}.

    \tbf{WTS}: $\mathfrak{B}^{\ast} = \{ e^{\mu} \}$ is a basis of $V^{\ast}$\footnote{Proving this statement gives our original claim immediately.}.
    \begin{itemize}
        \item[(1)] $\mathfrak{B}^{\ast}$ is linearly independent: let's start from $a_{\mu}e^{\mu} = 0$. Then
        \[ (v \in V) \quad a_{\mu} e^{\mu}(v) = 0 \]
        However, since $e^{\mu}$ is a linear map, it suffices to check their behavior at the basis of $V$, $e_{nu}$\footnote{This has deeper context in standard linear algebra course, and can be shown. However, we skip the detail.}. Hence
        \[ a_{\mu} e^{\mu}(e_{\nu}) = a_{\textcolor{red}{\mu}}\delta^{\textcolor{red}{\mu}}{}_{\nu} = a_{\nu} = 0 \]
        \item[(2)] $\left<\mathfrak{B}^{\ast}\right> = V^{\ast}$? For arbitrary $w \in V^{\ast}$, it can be expanded into
        \[ w = w(e_{1})e^{1} + w(e_{2})e^{2} + \cdots + w(e_{n})e^{n} := w_{\mu}e^{\mu} \]
    \end{itemize}
\end{proof}
\newpage

% ===== ===== ===== ===== ===== ===== ===== ===== ===== ===== ===== =====

\begin{remark}
    In the last line of proof, we defined the \tbf{component} of a dual vector, $w(e_{\mu}) = w_{\mu}$. Both the basis and component of dual vectors are closely related to the basis $e_{\mu}$ of original vector space $V$.
\end{remark}

In summary,
\begin{itemize}
    \item[-] Let $V$ be a $F$-vector space, and $v \in V$ be a vector. Then $v$ can be expanded into the linear combination of basis vectors.
    \[ \boxed{v = v^{\mu} e_{\mu}} \]
    \item[-] The concept of dual space naturally follows from the original vector space $V$. The dual vector space $V^{\ast} = \mathfrak{L}(V,W)$ has dual vectors (or covectors) as elements: $w \in V^{\ast}$ and $w \,:\, V \rightarrow F$. The dual vector can be expanded into dual basis.
    \[ \boxed{w = w_{\mu} e^{\mu}} \]
    where $w_{\mu} := w(e_{\mu})$.
\end{itemize}

\end{document}
